%! Author = vojmo
%! Date = 29.01.2022

\chapter{Analýza}

\section{Kvalitativní průzkum}

%TODO: Popsat více?
Nejprve bylo potřeba zjistit, co by potenciální uživatelé aplikace ocenili, tedy získat od nich požadavky.
Zvolili jsme kvalitativní průzkum, který se narozdíl od kvantitativního zamřuje na malou skupinu respondentů.
Pro získání odpovědí jsem si připravil sadu otevřených otázek, kterých jsem se držel při rozhovoru. Hovory byly
uskutečněny online a každy trval mezi dvaceti minutami a jednou hodinou.

Všechny rozhovory byly vedeny v rozmezí dvou týdnů a poté jsem z nich sestavil funkční a nefunkční požadavky.

\section{Funkční požadavky}

\subsection{F1: Správa receptů}
\subsection{F2: Sdílení receptů}
\subsection{F3: Objednávky přes služby typu rohlik.cz}
\subsection{F4: Plánovač}
\subsection{F5: Správa spíže}
\subsection{F6: Možnost ankety kolem jídelníčku}

\section{Nefunkční požadavky}

\subsection{U1: Dostupné jako webová aplikace}
\subsection{P1: Systém pro jednotky uživatelů}
\subsection{S1: Serverless s možností připojení na speciální API v budoucnu}

\section{Existující řešení}

%TODO: Citace
\subsection{Vareni.cz}

Na vareni.cz se nachází několik reklam, jsou velké a rušivé. Není zde možné si přidat soukromý recept a sdílet ho
pouze s vybranými uživateli. Celkový koncept přidání receptu je pouze veřejný, není tedy možné si zde vytvořit sbírku
oblíbených receptů z různých portálů.

%TODO: Citace
\subsection{toprecepty.cz}

Na tomto portálu byla bohužel nefunkční registrace, takže jsem nemohl nahlédnout na funkce poskytované přihlášeným
uživatelům. Opět zde byla přítomna velká reklama, která zabírala většinu stránky. Jinak byl web navrhnut přehledně,
ale narazil jsem na několik nefunkčních prvků na mobilním zobrazení. Zhodnotit přidání receptů a jak funguje jejich
sdílení zhodnotit nemohu, kvůli výše zmíněným problémům. Našel jsem funkci podobnou doporučování receptů, avšak se
pravděpodobně jedná o náhodné doporučení, které nemá nic společného s tím co má uživatel rád. Dále je na webu dostupný
online magazín, kde jsou různé články týkající se gastronomie.

%TODO: Citace
\subsection{recepty.cz}

I třetí zástupce existujících řešení používá reklamu přes celou stránku okolo jejího obsahu. Podobně jako toprecepty.cz
je zde magazín obsahující příspěvky na spoustu témat o vaření. U přidání receptu nebylo napsané, co se s receptem stane,
zda bude veřejný nebo se zobrazí pouze mě. Po kliknutí na tlačítko \uv{Uložit recept}, se zobrazila stránka s nadpisem
\uv{Recept čeká na schválení}. Nebylo tedy opět možné soukromé použití.

\subsection{Závěr}

Existující řešení na vyhledávání receptů nabízí pouze veřejné recepty a mají spoustu reklam. Moje řešení bude poskytovat
možnost soukromé sbírky receptů a jejich sdílení s vybranými uživateli či pomocí odkazu.
