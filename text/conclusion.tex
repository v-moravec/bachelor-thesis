%! Author = Vojta
%! Date = 25.11.2021

\chapter{Závěr}

% Shrnutí, možnosti do budoucna, nedoporučuje se používat emoční výrazy,
% nepoužívat obrázky, nadpisy, citace, nové myšlenky

Cílem práce bylo navržení a vytvoření webové aplikace, která usnadní manipulaci s recepty, surovinami či jejich nákupem.
Povedlo se naplnit většinu z původních požadavků, které stanovil vedoucí při zadávání a které jsem získal z průzkumu s potencionálními
uživateli. V první verzi aplikace, tedy prototypu, se dají využít základní funkce, které by měla aplikace poskytovat. Všechny tyto
funkce se dají v budoucnu rozšířit tak, aby splňovaly veškeré možnosti, které jsem v průběhu tvoření této práce objevil.

Ze začátku jsem se zabýval průzkumem konkurenčních řešení a kvalitativním průzkumem mezi potencionálními uživateli. Poté
jsem pokračoval návrhem papírových modelů a designem, což jsem spojil v jednu věc. Následovala volba technologií a tvorba
celého konceptu, jak vše bude fungovat. Mezitím jsem se domlouval s poskytovateli online nákupů surovin, zda by bylo možné
využít jejich služeb. Také jsem průběžně implementoval a přidával nové změny, které vzešly z problémů, na které jsem v průběhu
práce narazil. Nakonec jsem finální prototyp podrobil uživatelskému testování. Během tohoto testování jsem průběžně opravoval
chyby, na které testeři narazili.

Do budoucna se nabízí spousta možností okolo poskytnutých služeb od Rohlíku a Košíku. S tím také souvisí tvorba vlastního
API, které by aplikace pro tyto operace pravděpodobně potřebovala. Tím by se také osvobodila od různých limitů, které využívané
technologie s sebou přináší (potažmo verze, které jsou zdarma, po překročení limitů začíná být výhodnější platit pouze hosting
pro vlastní řešení). Ačkoliv při správné optimalizaci a chytrém využívání zdrojů, které momentální technologie nabízí, je možné
aplikaci rozšiřovat nad aktuální implementací.
