%! Author = Vojta
%! Date = 25.11.2021

\chapter{Implementace}
Tato kapitola popisuje tvoření protypu aplikace od založení projektu po vydání první verze. Zmíním zajímavé problémy,
které během implementace nastaly a jak jsem je řešil. Vše staví na předchozích kapitolách, tedy analýze, návrhu a použití
technologií, které jsem představil dříve.

\section{Založení projektu}
Před touto prací jsem neměl žádné zkušenosti se zakládáním větších projektů. Domluvil jsem se tedy s vedoucím a společně
jsme založili projekt pro \emph{Vue} aplikaci. Měli jsme již připravený \emph{Github} repozitář, ve kterém jsme projekt
verzovali.

Pro založení jsme využili \emph{vue ui}, což je grafické rozhraní pro správu \emph{Vue} projektů. Zde jsme zvolili konfiguraci
a přidali pluginy. Poté bylo potřeba některé z nich inicializovat.

\subsection{Firebase}
Nejdříve jsme založili projekt ve \emph{Firebase}. Stačí se přihlásit, otevřít konzoli a kliknout na tlačítko pro vytvoření projektu.
Poté se zadá název, popřípadě se projde konfigurací \emph{Google Analytics}. Rovnou jsme přešli na \emph{Blaze plan}, díky kterému
se otevřelo více možností. Je ale potřeba si kontrolovat, že aplikace nepřesáhne žádné limity~\cite{FirebaseLimits}, jinak se strhne
částka ze zadané platební karty.

Po založení se do projektu přidá aplikace. Díky tomu \emph{Firebase} vygeneruje kód, který se musí do aplikace přidat.


\begin{listing}
    \caption{Zbytečný kód}\label{list:8-6}
    \begin{minted}{C}
    #include<stdio.h>
    #include<iostream>
    // A comment
    int main(void)
    {
        printf("Hello World\n");
        return 0;
    }
    \end{minted}
\end{listing}


\section{Struktura projektu}

\section{Tvoření komponent}
