%! Author = Vojta
%! Date = 21.10.2021

\chapter{Technologie}

\section{Výběr}
Vzhledem k tomu, že aplikace byla původně zamýšlena jako osobní projekt, který by si následně spravoval sám vedoucí a já
jsem měl zkušenosti pouze s frameworkem Vue.js, hlavní technologie, okolo které se projekt postaví, byla předem daná. Poté
jsem postupně vybíral další součásti, které bychom mohli využít. Velkou výhodou bylo, že právě vedoucí práce má s většinou
z těchto knihoven či pluginů zkušenosti, a tak když se mi něco nedařilo, mohl jsem se na něj obrátit.

\section{Vue.js}
Vue je progresivní JavaScriptový framework, který narozdíl od konkurenčních řešení (React, Angular) nezaštiťuje žádná velká
korporace, ale je vyvíjen komunitou.\cite{VueJS} Zvolili jsme verzi 3, protože je to lepší řešení do budoucna, než později aktualizovat
celou aplikaci z verze 2. % TODO

\section{Vuetify}
Vuetify je knihovna implementující různé komponenty, které je možné použít při tvorbě uživatelského rozhraní. Kromě toho
usnadňuje práci s rozložením na stránce, přizpůsobením barevného téma, ikonami atd. Další výhodou jsou týdenní aktualizace
momentální verze, které přidávají nové funkce a opravují nalezené chyby.\cite{VuetifyWhy}
Bohužel tato knihovna nebyla v době vývoje plně dokončena a byla pouze v alpha verzi, tudíž spoustu věci nefungovalo jak by mělo.
% TODO

\section{Vuex}
% TODO

\section{Vue i18n}
% TODO

\section{Vue Router}
% TODO

\section{Firebase}
% TODO
