%! Author = Vojta
%! Date = 21.10.2021

\chapter{Technologie}

\section{Výběr}
Vzhledem k tomu, že aplikace byla původně zamýšlena jako osobní projekt, který by si následně spravoval sám vedoucí a já
jsem měl zkušenosti pouze s frameworkem Vue.js, hlavní technologie, okolo které se projekt postaví, byla předem daná. Poté
jsem postupně vybíral další součásti, které bychom mohli využít. Velkou výhodou bylo, že právě vedoucí práce má s většinou
z těchto knihoven či pluginů zkušenosti, a tak když se mi něco nedařilo, mohl jsem se na něj obrátit.

\section{Vue.js}
Vue je progresivní JavaScriptový framework, který narozdíl od konkurenčních řešení (React, Angular) nezaštiťuje žádná velká
korporace, ale je vyvíjen komunitou.\cite{VueJS} Zvolili jsme verzi 3, protože je to lepší řešení do budoucna, než později aktualizovat
celou aplikaci z verze 2. V pozdější fázi vývoje se ale ukázalo, že pro verzi 3 nebyla plně dokončena hlavní knihovna, kterou jsem chtěl využít.
Musel jsem tak ponížit verze všech závislostí a přejít tak na verzi 2.

\section{SPA}
SPA neboli \emph{single page application} je stránka, která využívá takové architektury, kde použitá technologie nejen kontroluje
vzhled stránky, ale i data a manipulaci s nimi a navigaci tím způsobem, že není nutné provádět obnovení stránky.\cite{VueSPA}

\section{Vuetify}
Vuetify je knihovna implementující různé komponenty, které je možné použít při tvorbě uživatelského rozhraní. Kromě toho
usnadňuje práci s rozložením na stránce, přizpůsobením barevného téma, ikonami atd. Další výhodou jsou týdenní aktualizace
momentální verze, které přidávají nové funkce a opravují nalezené chyby.\cite{VuetifyWhy}
Bohužel tato knihovna nebyla v době vývoje plně dokončena a byla pouze v alpha verzi, tudíž spoustu věci nefungovalo jak by mělo.
% TODO

\section{Vuex}
Knihovna Vuex se využívá pro uložení stavu aplikace. Je potřeba využít u dat, která jsou využívána na více místech a jejich
předávání skrz komponenty by bylo jinak složité.

\section{Vue i18n}
I18n je rozšíření pro překlady, díky kterému je možné texty v aplikaci napsat v několika jazycích. Nejprve jsem tvořil aplikaci dvojjazyčně
v angličtině a čestině, ale nakonec jsem se rozhodl, že prozatím dává aplikace smysl pouze v češtině. Nicméně překlady jsem ponechal a v budoucnu
je možné je využít.

\section{Vue Router}
V aplikaci je potřeba mít navigaci. Ve Vue se používá Vue Router, který umožní pohyb po různých stránkách.

\section{Firebase}
Firebase jsem použil na backendu. To mi umožnilo mít vše na jednom místě. Uložiště, databázi, autorizaci uživatelů, hosting atd.

\subsection{Firestore}
Cloud Firestore je NoSQL dokumentová databáze. Narozdíl od SQL databází, které se soustředí na snížení duplikace dat, se dokumentová
databáze zaměruje na časté aktualizace a změny. Největší rozdíl mezi těmito typy je způsob uložení dat. SQL reprezentuje data pomocí
tabulek s řádky a sloupci, dokumentová databáze má JSON dokumenty a jejich kolekce. To vede k flexibilnějšímu datovému modelu, rychlejším
dotazům a lehčímu vývoji pro vývojáře.

Firestore nabízí vlastní řešení bezpečnosti přes Firebase Rules.

% TODO: https://www.mongodb.com/nosql-explained/nosql-vs-sql
% TODO: https://firebase.google.com/products/firestore?gclid=CjwKCAjw9LSSBhBsEiwAKtf0n4LxN1LXFGXdwYILgot1F2Sm9j5vVf6RI4KViRZuZDt6rrnohROUlRoCML8QAvD_BwE&gclsrc=aw.ds

\section{Fulltextové vyhledávání}

Při práci s Firebase jsem zjistil, že při dotazování se na záznamy není možné filtrovat podle názvu
(abych byl přesný, možné to je, ale není to vhodné). Po zkoumání dokumentace, jsem našel stránku s doporučením pro
fulltextové hledání. V nabídce byla tři řešení. % TODO: Cite https://firebase.google.com/docs/firestore/solutions/search

\begin{itemize}
    \item Elastic
    \item Algolia
    \item Typesense
\end{itemize}

Problém jsem řešil s vedoucím a přišli jsme na několik možných řešení sami. Stáhnout si data o všech receptech ve formátu
\emph{id:name} a poté filtrovat výsledky hledání na FE. Dále bychom mohli použít Firebase Cloud Functions, kde bychom
využili hashování. Nakonec jsme se ale rozhodli, že využijeme jedno z nabízených řešení přímo Googlem.

Vybrali jsme Algolii, kvůli dobré podpoře Vue a Firebase, nejmenší složitosti implementace a bezplatnému základnímu plánu.

Při vývoji jsem ale zjistil, že základní režim (tedy 10 000 čtení za měsíc) nejspíše stačit nebude. Nakonec jsem proto přešel na
frontendové vyhledávání. Všechna data jsem stáhnul při načtení aplikace a poté s nimi dál pracoval.

\section{PWA}
