%! Author = Vojta
%! Date = 21.10.2021

\chapter{Nákup surovin}

\section{Možnosti}
Vedoucí práce již dříve používal aplikaci Zdravý stůl~\cite{ZdravyStul}. Tam bylo možné si jídlo objednat přes rohlik.cz (vložit do košíku).
Tudíž jsme chtěli tuto funkcionalitu zachovat a přidat možnosti jako například vytvoření objednávky u konkurence - kosik.cz
či zobrazení interaktivního nákupního listu.

\section{Komunikace s Rohlíkem a Košíkem}
Nejdříve jsme se rozhodli kontaktovat Rohlík. Po pár vyměněných e-mailech jsme obdrželi celou dokumentaci k jejich API,
které nám otevřelo spoutu možností i do budoucna. Například bychom mohli sledovat, jaké suroviny jsou právě ve slevě a
podle toho doporučovat jídla.

Od Košíku jsme dostali pozvání na schůzku, kde jsme si mohli prohlédnout i jejich kanceláře. Na schůzce jsme hned na začátku
zjistili, že API pro partnery narozdíl od Rohlíku ještě dostupné není (Rohlík na API pro partnery také teprve pracuje,
ale zatím jsme dostali přístup k API pro jejich aplikaci), ale už na něm pracují. Plánované období vydání je
první kvartál roku 2022. Pro jeho využití je však potřeba OAuth server, který zatím není dostupný. % TODO: Zjistit v lednu.
Jako alternativu jsme dohromady vymysleli link na přidání surovin přímo do košíku, ze kterého nakonec sešlo, protože jsme
poté objevili funkci \uv{Nákupní lístek}, kterou bychom mohli využít. Odeslali bychom seznam surovin a uživatel by si je poté
mohl vybrat přímo z nabídky na Košíku. Nakonec jsme zjistili, že by se Košíku hodilo rozrůst sbírku receptů a bylo by možné
pro uživatele naší aplikace nabínout jejich recepty Košíku, který by je následně odkoupil.
