%! Author = Vojta
%! Date = 25.11.2021

\chapter{Závěr}

% Shrnutí, možnosti do budoucna, nedoporučuje se používat emoční výrazy,
% nepoužívat obrázky, nadpisy, citace, nové myšlenky

Cílem práce bylo navržení a vytvoření webové aplikace, která usnadní manipulaci s recepty, surovinami či jejich nákupem.
Povedlo se naplnit všechny původní požadavky a díky průzkumu přidat další užitečné funkce.

Ze začátku jsem se zabýval průzkumem konkurenčních řešení, kvalitativním průzkumem mezi potencionálními uživateli. Poté
jsem pokračoval návrhem papírových modelů a designem, což jsem spojil v jednu věc. Následovala volba technologií a tvorba
celého konceptu jak vše bude fungovat. Mezitím jsem se domlouval s poskytovateli online nákupů surovin, zda by bylo možné
využít jejich služeb. Také jsem průběžně implementoval a přidával nové změny, které vzešly z problémů, na které jsem v průběhu
práce narazil. Nakonec byl konečný prototyp podroben uživatelskému testování.

Do budoucna se nabízí spousta možností okolo poskytnutých služeb od Rohlíku a Košíku. S tím také souvisí tvorba vlastního
API, které by aplikace pro tyto operace určitě potřebovala. Tím by se také osvobodila od různých limitů, které využívané
technologie s sebou přináší (potažmo verze které jsou zdarma, po překročení limitů začíná být výhodnější platit pouze hosting
pro vlastní řešení).
