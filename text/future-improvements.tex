%! Author = Vojta
%! Date = 25.11.2021

\chapter{Možnosti aplikace v budoucnosti}

Během vývoje nastaly různé změny a některé funkce byly moc složité na implementaci nebo nebyly nutné v první verzi.
V této kapitole nastíním, jakým směrem by se aplikace mohla vydat.

\section{Interakce mezi uživateli}
V prototypu je možné pouze vytvořit skupinu, pozvat do ní ostatní uživatele a přidat do ní recepty. Původně jsem zamýšlel
i textovou komunikaci, ale nakonec jsme se s vedoucím shodli, že zatím není potřeba. Ovšem do budoucna by se mohla takováto
funkce hodit, v kombinaci s komentáři a hodnocením receptů. Poté by uživatelé mohli vybírat mezi veřejnými recepty podle
jejich kvality a sdílet své pocity či vychytávky.

Na to by se dalo navázat vytvořením profilů uživatelů. Na tomto profilu by uživatelé mohli přidávat příspěvky jako je tomu
například na Instagramu. Na této síti jsou populární krátká videa s recepty a tím pádem je zde potenciál pro případný % TODO: Citace
příliv uživatelů.

\section{Časovač}
U návrhu zobrazení receptu na mobilním zařízení jsem přidal časovač. V receptu by mohla být tlačítka u daných kroků, která
by spustila časovač na limit, který byl stanoven při vytvoření. Například by tedy po deseti minutách začal zvonit telefon, aby
uživatele upozornil na uvařené špagety. Seznam těchto měření by byl dostupný jako plovoucí tlačítko
v celé aplikaci a uživatel by tak k vaření potřeboval pouze mobilní zařízení.

\section{Verze FIT}
Kromě normálních receptů, by se dala aplikace přepnout do fitness módu, kde by zobrazovala pouze zdravé recepty. Nabídla by
také sledování kalorií, plány pro zlepšení stravy či jiné návrhy na úpravu jídelníčku.

\section{Pohodlí pro uživatele}
Jak jsem zmínil ve výsledcích kvalitativního průzkumu, uživatelé mají často problémy se zašpiněním dipleje na jejich zařízení při vaření.
Zabránit nepotřebného kontaktu se zařízením by mohlo pomoci například vynucení, aby obrazovka nezhasnula. Bylo by potřeba prozkoumat
\emph{Screen Wake Lock API} nebo nějaký plugin třetí strany. % TODO: Citace https://developer.mozilla.org/en-US/docs/Web/API/Screen_Wake_Lock_API

\section{Propojení se službami Košík a Rohlík}
Tato práce měla být původně rozdělena na dvě, na frontend a na backend. Nakonec se nepodařilo sehnat studenta na backendovou část,
a tak jsem musel zpracovat fullstack aplikaci. Nezbyl tedy čas na implementaci doporučování, jednoduché přidání do košíku a dalších
funkcí napojené na služby Rohlíku a Košíku.

V budoucnu je ale jistě priorita tyto funkce zpřístupnit. Doporučení receptů by šlo založit na aktuálních slevách, daném ročním období či
uživatelově nákupní historii. S nákupy se pojí automatické přidání do spíže a sledování počtu surovin. Toto rozšíření spočívá v přidání
privátních ingrediencí, ke kterým si uživatel nebo skupina může přiřadit vlastní odkazy na zmíněné služby a počet kusů.

\section{Skupiny}
Při pozvání se nekontroluje žádný časový úsek či jiné ověření zda je pozvánka platná. Přidáním tokenu by se dala omezit doba platnosti
a tím zabránit nechtěnému šíření. Také stojí za zvážení operace nad členy skupiny. Tedy odebrání členů či smazání skupiny, které je zatím
nemožné.
