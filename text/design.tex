%! Author = vojmo
%! Date = 29.01.2022

\chapter{Návrh}

\section{Návrh vzhledu aplikace}

Při navrhování aplikace se většinou začíná s takzvanými wireframy. Wireframe je rozložení prvků aplikace, které ještě
nemají žádný vzhled. Jsou to například tlačítka, sekce pro různý obsah atd. Já jsem zvolil lehce odlišný přístup, tedy
vytvoření wireframu již s designem. K této tvorbě jsem použil nástroj Adobe XD~\cite{AdobeXD}.

U vzhledu jsem se inspiroval moderními operačními systémy jako např. Windows 11~\cite{Windows11}, One UI~\cite{OneUI}
(nadstavba Android od společnosti Samsung) atd.
Chtěl jsem dosáhnout jednoduchého, přehledného designu, který bude pro uživatele příjemný na používání. Postupně jsem se
propracoval přes několik verzí a ve výsledku jsem zvolil průhledný styl podkladu komponent s \emph{mesh gradientem} na pozadí.

Již od počátku jsem bral v potaz rozdíly mezi zobrazením na mobilu a počítači, díky čemuž jsem mohl design uzpůsobit
pro více typů obrazovek. Začal jsem vytvářením úvodní obrazovky, na které se zobrazí rychlé odkazy na nejdůležitější části
aplikace, vyhledávání a logo. Tato stránka by měla být přehledná a čistá, proto jsem zvolil ikonky, které vystihují dané
odkazy. Díky tomu se při zobrazení na mobilu se tak mohou skrýt pomocné texty.

Dále jsem pokračoval se zobrazením receptů. Na verzi pro počítač jsem na levé straně umístil filtrování a zbylou plochu jsem ponechal
pro samotné recepty. Na menších obrazovkách se filtrování přesune nad seznam receptů a již nebude viditelné při posunutí směrem dolů.
Na \uv{karty} s recepty jsem vybral nejdůležitější informace. Tedy název, štítky a obrázek receptu. Nejprve jsem přidal do spodní části
karty i ozdobný motiv, později při implementaci jsme se ale s vedoucím dohodnuli, že pouze zabírá místo a bude lepší jej odebrat.

Zobrazení samotného receptu je na velkých a malých obrazovkách velmi odlišné. Zatímco na počítači se uživateli zobrazí všechny
informace v několika oddělených obdélnících, na mobilu se každá z těchto částí rozdělí na samostatnou záložku. Také se na spodku
obrazovky zobrazí lišta, pomocí které je možné záložky přepínat.

Seznam ingrediencí je stejný jako u receptů. Pro zobrazení ingredience platí totéž.

Plánovač jsem navrhnul jako kalendář, do kterého se dají přesunout recepty pomocí \emph{drag and drop}. Pro rychlý návrh receptu
jsem zvolil schéma podobné dnešním seznamkám. Tedy přijmutí popřípadě odmítnutí navrženého receptu je po stranách zobrazené karty
a vlevo ze nachází zásobník již zvolených receptů.

\section{Návrh loga}

\section{Ukázky}

\section{Databáze ve Firebase}
Jak jsem již zmínil, Firestore% TODO Odkaz

\section{Uložiště ve Firebase}

\section{Data v aplikaci}

\section{Funkce aplikace}
