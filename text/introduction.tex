%! Author = Vojta
%! Date = 21.10.2021

\chapter{Úvod}

Vaření se týká spousty lidí. Většina k tomu používá recepty, které hledají na internetu. Bohužel recepty, které najdou
a které se jim osvědčí, jsou často na jiných stránkách a správa těchto receptů se stává nereálná.

Výsledek této práce bude moct použít široká veřejnost -- každý kdo vaří podle receptů. Aplikace jim pomůže s jednoduchým
zadání receptů do sbírky, kde mohou mít všechny recepty na jednom místě, nákupem surovin, plánováním či správou toho všeho.

Téma mě zaujalo, protože si myslím, že nabízí spoustu míst, pro učení se nových věcí. Také jsem nenašel žádnou alternativu,
která by nabízela všechny funkce, které moje řešení poskytuje.

V práci popisuji proces vývoje webové aplikace od analýzy přes návrh až po implementaci. Zaměřím se hlavně na frontend v
javascriptovém frameworku Vue.js, ale provedu čtenáře i backendovou částí, která je tvořena pomocí nástrojem Firebase od
společnosti Google.

\section{Cíl}
Cílem je vytvořit prototyp webové aplikace pro správu receptů, která zároveň usnadňí nákup surovin.

Nejdříve zanalyzuji potřeby potenciálních uživatelů, kde se pokusím zjistit, které všechny funkce by ocenili. Poté se
zaměřím na existující řešení, která porovnám s mým řešením a popíšu vylepšení.

Dále vytvořím návrh designu aplikace -- wireframy, které zachycují zároveň, jak aplikace vypadá. Důraz kladu na rozdíly
mezi použitím na mobilu a počítači.

Na základě analýzy a návrhu vyberu technologie, které použiju pro implementaci funkčního prototypu.

Na konec podrobím aplikaci uživatelskému testování a zhodnotím jeho výsledky.

\section{Struktura}
