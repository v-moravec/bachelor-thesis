%! Author = Vojta
%! Date = 21.10.2021

\chapter{Úvod}

Vaření se týká spousty lidí. Mladší generace k tomu používá recepty, které jsou dostupné na internetu. Recepty, které najdou
a které se jim osvědčí, jsou často na jiných stránkách a správa těchto receptů se stává nereálná. Na druhou stranu na nejznámějších
portálech není možné si recepty ukládat soukromně a tak spoustu starších lidí nevyužije možnosti digitalizace jejich receptů.
Většina z nich si totiž uchovává své recepty napsané na listech papíru, což ale v dnešní době již není praktické řešení. Už jen kvůli
sdílení receptu s někým, kdo si ho vyžádá, což je celkem běžná záležitost.

Výsledek této práce bude moci použít široká veřejnost -- každý kdo vaří podle receptů. Aplikace jim pomůže s jednoduchým
zadání receptů do sbírky, kde mohou mít všechny recepty na jednom místě. Dále budou moci uživatelé využít plánovače, kam
lze zanést již přidané recepty či počet porcí. Pokud nejsou zvyklí plánovat a spíše řeší věci na poslední chvíli, bude
pro ně připraven rychlý návrh receptu. V neposlední řadě bude jednoduché své recepty sdílet s ostatními uživateli.

V práci popisuji proces vývoje webové aplikace od analýzy přes návrh až po implementaci. Zaměřím se hlavně na frontend v
javascriptovém frameworku Vue.js, ale provedu čtenáře i backendovou částí, která je tvořena pomocí nástrojem Firebase od
společnosti Google. Začnu sběrem informací od potencionálních uživatelů, pomocí kterých sestavím funkční a nefunkční požadavky.
Poté zanalyzuji konkurenční řešení a popíšu jejich plusy či nevýhody. Poté představím technologie, které jsem si vybral pro
vývoj. Dále navrhnu uživatelské rozhraní, grafické prvky či datovou strukturu aplikace. Na základě předchozích kapitol implementuji
aplikaci a zmíním problémy, se kterými jsem se při vývoji setkal. Sestavím test pro uživatele a zhodnotím jeho výsledky.
Na závěr zmíním funkce které jsem nestihl implementovat či rozšíření, která by se v budoucnosti hodila.

Téma mě zaujalo, protože si myslím, že nabízí spoustu míst, pro učení se nových věcí. Také jsem nenašel žádnou alternativu,
která by nabízela všechny funkce, které moje řešení poskytuje.

\chapter{Cíl}
% TODO: Vylepšit
Cílem této bakalářské práce je vytvořit prototyp webové aplikace pro správu receptů. Dále je nutné prozkoumat možnosti
zjednodušení nákupu surovin či plánování vaření jídel.

Nejdříve zanalyzuji potřeby potenciálních uživatelů, kde se pokusím zjistit, které všechny funkce by ocenili. Poté se
zaměřím na existující řešení, která porovnám s mým řešením a popíšu vylepšení.

Dále vytvořím návrh designu aplikace -- wireframy, které zachycují zároveň, jak aplikace vypadá. Důraz kladu na rozdíly
mezi použitím na mobilu a počítači.

Na základě analýzy a návrhu vyberu technologie, které použiju pro implementaci funkčního prototypu.

Na konec podrobím aplikaci uživatelskému testování a zhodnotím jeho výsledky.
